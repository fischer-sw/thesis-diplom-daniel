\documentclass[../thesis.tex]{subfiles}
\begin{document}

\chapter{Discussion}
\label{chp: dicus}
All numerical investigations do have some limitations. Assumptions that are made and simplification that have been done do set boundaries. The limitations are discussed in this section.

The first limitation is, that the model is simulating a 2D case. The assumption has been made, that all cases show axisymmetric behaviour so only a 2D slice of the reactor is modelled. In the real world there are always some 3D effects, changing the front's behaviour compared to the simulation model. One of these effects could be mass transfer in tangential direction. Adding to the axisymmetric assumption, the homogenous velocity profile set as the inlet's boundary condition is impossible to meet in a real world experiment. Since in every Hele-Shaw cell the fluid injected has to come from somewhere, the velocity at the inlet is not homogeneous in it's magnitude and direction. This poses a large influence on the front's distortion at early times. 

Other influences that can be observed under real world conditions, are different thermophysical properties of the species taking part in the reaction. Since two reactant molecules form one molecule of product they all have to have different densities, viscosities and diffusion coefficients. Within the model the reaction speed can be set to a very high value, but under normal conditions the reaction speed might have a lower value. The more crucial part, when looking at the reaction is, that there is no backwards reaction. Real world reaction systems always consist of a forward and backward reaction. The reaction could also have an impact on the temperature, lowering it in case of an endotherm one or raising it if it is an exothermal one. In addition to the reaction changing the temperature the reactants themself induce thermal mass into the system, affecting the temperature.

Within the model the influence of gravity can be turned of completely. Under normal conditions the gravitational effects can only be lowered but not get rid off. The only way to achieve 0g conditions in real world would be to do experiments at Lagrange points in space, like the one the James-Webb-Telescope is located at. Since that is not practically viable there will always be gravitational influences. With gravity always being there in the real world, the fronts might change in shape, due to density differences, as schematically shown in \autoref{fig: 0g_example}.

In addition to the different thermophysical properties, the initial concentrations do influence especially the amount of product build. The influence of the ratio of the reactant's concentrations was investigated in \cite{comolli2021dynamics} and shows an influence on all front metrics.

The front positions are mainly influenced by the Peclet number. Even higher Peclet numbers will probably do not result in a changing behaviour, if laminar conditions are still met.

The front width is influenced by Peclet and Schmidt number. Running the simulations for longer durations might answer the question, if the long-term regime of constant width is met or if something else changes at way later times. Most interest is here in the cases with the higher gap widths.
The formed product shows the most interesting behaviour for the small Schmidt number. Lowering the Schmidt number even more might see an even bigger change in production rate.


\chapter{Conclusion and Outlook}
\label{chp:out_con}
In this work a numerical study was performed for the reaction of $A +B \rightarrow C$ within a radial reactor. The numerical model was implemented using the software \texttt{ANSYS FLUENT}. Different geometries and conditions were investigated regarding the evolving front's shape, width and amount of product formed. The model validation was performed successfully using existing experimental data.
The implementation and post processing is setup in a way that can easily be modified for the existing geometry or be applied to other models created in \texttt{ANSYS FLUENT} that are similar to the one used in this work. The model delivers more insight into the phenomenon of reaction-diffusion-advection fronts and provides information otherwise not accessible to experimental setups by giving a cross section view of the reactor.
The change in reactor gap height has no significant influence on the front positions since the results for all geometries show similar behaviour. The front widths do show different behaviour under different gap heights. For the Schmidt number of 12000 all cases behave similarly with a fast initial growth trending towards a final constant value. The cases with a lower Schmidt number of 2430 do show significant changes for different reactor geometries. The visible peak in all plots do shift in position and sharpness.  
The total amount of product formed shows comparable results for all geometries. An early growth in product is followed by a decline towards a constant production rate. This production rate is mainly influenced by the Peclet number. For the highest gap height the production rate seems to mainly be inside the initial growing phase. For the lower gap heights the production rate declines at the end showing that the rate is in the process of transitioning towards the long-term constant value.
The raw data generated by the simulation certainly do hide more valuable information. Further analysis could be done to the front's shape as done in \cite{perez2019upscaling, villermaux2012mixing}. 
In addition to the already performed cases, that do have the some inlet concentrations for species $A$ and $B$, the concentration ratio might be an interesting topic to look further into or the inlet's conditions could be varied in time. There are still more influences that might prove valuable to investigate in the future. 
The possibility of running cases in parallel could help reduce the amount of experiments needed in the future. The model could also be used for validating existing experimental runs with different conditions as long as the current limitations are thought of. Since experiments under 0g conditions are monetary expensive and consume a lot of time for preparation the developed model provides a resource efficient alternative.

The developed model sets a step in the direction to gain further understanding on what is happening at the front's formation process at early time stages under 0g conditions. A foundation is laid in this work that can be used for further research and investigations.

\end{document}