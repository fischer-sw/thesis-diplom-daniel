\documentclass[../thesis.tex]{subfiles}
\begin{document}

Conti equation

\begin{equation}
	\label{eqn:ansys_conti}
	\dfrac{\partial}{\partial x} (\rho u_x) + \dfrac{\partial }{\partial r} (\rho u_r)
	+ \dfrac{\rho u_r}{r} = 0
\end{equation}

radial momentum equation
\begin{gather}
	\label{eqn: rad_mom}
	\dfrac{1}{r} \dfrac{\partial}{\partial x}(r \rho u_x^2)
	+ \dfrac{1}{r} \dfrac{\partial}{\partial r}(r \rho u_r u_x) = 
	- \dfrac{\partial p}{\partial x} + \dfrac{1}{r} \dfrac{\partial }{\partial x} \left[ 
	r \mu \left( 2 \dfrac{\partial u_x}{\partial x} - \dfrac{2}{3}(\nabla \cdot \mathbf{u}) \right)
	\right] + \\ \nonumber
	\dfrac{1}{r} \dfrac{\partial }{\partial r} \left[ 
	r \mu \left( \dfrac{\partial u_x}{\partial r} - \dfrac{\partial u_r}{\partial x} \right)
	\right]	
\end{gather}

axial momentum equation

\begin{gather}
	\label{eqn: axi_mom}
	\dfrac{1}{x} \dfrac{\partial}{\partial r}(r \rho u_x u_r)
	+ \dfrac{1}{r} \dfrac{\partial}{\partial x}(r \rho u_r^2) = 
	- \dfrac{\partial p}{\partial r} + \dfrac{1}{r} \dfrac{\partial }{\partial x} \left[ 
	r \mu \left( \dfrac{\partial u_r}{\partial x} - \dfrac{\partial u_x}{\partial r} \right)
	\right] \\ \nonumber
	+ \dfrac{1}{r} \dfrac{\partial }{\partial r} \left[ 
	r \mu \left( 2 \dfrac{\partial u_r}{\partial r} - \dfrac{2}{3}(\nabla \cdot \mathbf{u}) \right)
	\right] -
	2 \mu \dfrac{u_r}{r^2}+ \dfrac{2}{3} \dfrac{\mu}{r}(\nabla \cdot \mathbf{u})
\end{gather}

species distribution equation

\begin{equation}
	\dfrac{\partial c_A}{\partial t} + \left( \dfrac{\partial u_x}{ \partial x} + \dfrac{\partial u_r}{ \partial r} + \dfrac{u_r}{r} \right) c_A = D \left( \dfrac{\partial^2 c_A} {\partial r^2} + \dfrac{\partial^2 c_A}{\partial x^2} \right)  - kc_A c_B
\end{equation}

\begin{equation}
	\dfrac{\partial c_B}{\partial t} + \left( \dfrac{\partial u_x}{ \partial x} + \dfrac{\partial u_r}{ \partial r} + \dfrac{u_r}{r} \right) c_B =  D \left( \dfrac{\partial^2 c_B} {\partial r^2} + \dfrac{\partial^2 c_B}{\partial x^2} \right) - kc_A c_B
\end{equation}

\begin{equation}
	\dfrac{\partial c_C}{\partial t} + \left( \dfrac{\partial u_x}{ \partial x} + \dfrac{\partial u_r}{ \partial r} + \dfrac{u_r}{r} \right) c_C =  D \left( \dfrac{\partial^2 c_C} {\partial r^2} + \dfrac{\partial^2 c_C}{\partial x^2} \right) + kc_A c_B
\end{equation}

energy equation
\begin{equation} 
	\frac{\partial}{\partial t} (\alpha_q \rho_q h_q ) + \nabla \cdot (\alpha_q \rho_q \vec u_q h_q )   = \alpha_q \frac{\partial p_q}{\partial t} + \overline{\overline{\tau}}_q : \nabla \vec u_q
\end{equation}

\begin{equation}
	\label{eqn: reaction_detail}
	A + B \xrightarrow{r} C
\end{equation}

The reaction rate $r$ can be calculated using \autoref{eqn: reaction_vel}. From this equation it can be seen that the reaction is of second order and is dependent of the species $A$ and $B$.

\begin{equation}
	\label{eqn: reaction_vel}
	r = k \cdot c_A \cdot c_B
\end{equation}

The reaction velocity equation uses the rate constant $k$ which can be computed using the context of \autoref{eqn: reaction_rate_detail}.

\begin{equation}
	\label{eqn: reaction_rate_detail}
	k = k_{\infty} \cdot e^{- \frac{E_A}{R \cdot T}}
\end{equation}

\begin{equation}
	k_{\infty} = 1 \cdot 10^{15} \left[ \frac{\text{l}^2}{\text{mol} \cdot \text{s}} \right]
\end{equation}

\begin{equation}
	E_A = 1 \cdot 10^4  \left[ \frac{\text{J}}{\text{mol}} \right]
\end{equation}

\end{document}S