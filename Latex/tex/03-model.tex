\documentclass[../thesis.tex]{subfiles}
\begin{document}

\chapter{Simulation model}
\label{chp:model}

Within this part the creation process of the simulation model will be explained in greater detail. At first the three basic equations the software solves will be given followed by a brief overview of the steps taken to create the model. Then the model's results are explained and compared to already done experiments for validation.

\section{governing equations}
\label{sec:gov_eqn}
To obtain the results of interest the model needs to solve three different equations that are linked to each other.

\subsection{mass conservation}
To gain information about the velocity field the mass conservation equations for mass and momentum needs to solved. For 2D axisymmetric case as used here the equations looks like this \cite{manual2009ansys}:

\begin{equation}
	\label{eqn:ansys_conti}
	\dfrac{\partial \rho}{\partial t} + \dfrac{\partial}{\partial x} (\rho u_x) + \dfrac{\partial }{\partial r} (\rho u_r)
	+ \dfrac{\rho u_r}{r} = S_m
\end{equation}

This is the continuity equation where $r$ is the radial coordinate, $u_x$ is the axial velocity, $u_r$ is the radial velocity, $\rho$ is the density and $S_m$ is the source term. In addition to the shown continuity equation the radial and axial momentum equations need to be solved.

%\begin{equation}
%	\dfrac{\partial}{\partial t}(\rho u_x) + \dfrac{1}{r} \dfrac{\partial}{\partial x}(r \rho u_x^2)
%	+ \dfrac{1}{r} \dfrac{\partial}{\partial r}(r \rho u_r u_x) = 
%	- \dfrac{\partial p}{\partial x} + \dfrac{1}{r} \dfrac{\partial }{\partial x} \left[ 
%		r \mu \left( 2 \dfrac{\partial u_x}{\partial x} - \dfrac{2}{3}(\nabla \cdot \mathbf{u}) \right)
%	\right] + \dfrac{1}{r} \dfrac{\partial }{\partial r} \left[ 
%	r \mu \left( \dfrac{\partial u_x}{\partial r} - \dfrac{\partial u_r}{\partial x} \right)
%	\right]
%\end{equation}

\begin{gather}
	\dfrac{\partial}{\partial t}(\rho u_x) + \dfrac{1}{r} \dfrac{\partial}{\partial x}(r \rho u_x^2)
	+ \dfrac{1}{r} \dfrac{\partial}{\partial r}(r \rho u_r u_x) = 
	- \dfrac{\partial p}{\partial x} + \dfrac{1}{r} \dfrac{\partial }{\partial x} \left[ 
	r \mu \left( 2 \dfrac{\partial u_x}{\partial x} - \dfrac{2}{3}(\nabla \cdot \mathbf{u}) \right)
	\right] + \\ \nonumber
	\dfrac{1}{r} \dfrac{\partial }{\partial r} \left[ 
	r \mu \left( \dfrac{\partial u_x}{\partial r} - \dfrac{\partial u_r}{\partial x} \right)
	\right]	
\end{gather}

%\begin{equation}
%	\dfrac{\partial}{\partial t}(\rho u_r) + \dfrac{1}{x} \dfrac{\partial}{\partial r}(r \rho u_x u_r)
%	+ \dfrac{1}{r} \dfrac{\partial}{\partial x}(r \rho u_r^2) = 
%	- \dfrac{\partial p}{\partial r} + \dfrac{1}{r} \dfrac{\partial }{\partial x} \left[ 
%	r \mu \left( \dfrac{\partial u_r}{\partial x} - \dfrac{\partial u_x}{\partial r} \right)
%	\right]
%	+ \dfrac{1}{r} \dfrac{\partial }{\partial r} \left[ 
%	r \mu \left( 2 \dfrac{\partial u_r}{\partial r} - \dfrac{2}{3}(\nabla \cdot \mathbf{u}) \right)
%	\right] -
%	2 \mu \dfrac{u_r}{r^2}+ \dfrac{2}{3} \dfrac{\mu}{r}(\nabla \cdot \mathbf{u}) + \rho \dfrac{u_z^2}{r} + %F_r
%\end{equation}

\begin{gather}
	\dfrac{\partial}{\partial t}(\rho u_r) + \dfrac{1}{x} \dfrac{\partial}{\partial r}(r \rho u_x u_r)
	+ \dfrac{1}{r} \dfrac{\partial}{\partial x}(r \rho u_r^2) = 
	- \dfrac{\partial p}{\partial r} + \dfrac{1}{r} \dfrac{\partial }{\partial x} \left[ 
	r \mu \left( \dfrac{\partial u_r}{\partial x} - \dfrac{\partial u_x}{\partial r} \right)
	\right] \\ \nonumber
	+ \dfrac{1}{r} \dfrac{\partial }{\partial r} \left[ 
	r \mu \left( 2 \dfrac{\partial u_r}{\partial r} - \dfrac{2}{3}(\nabla \cdot \mathbf{u}) \right)
	\right] -
	2 \mu \dfrac{u_r}{r^2}+ \dfrac{2}{3} \dfrac{\mu}{r}(\nabla \cdot \mathbf{u}) + \rho \dfrac{u_z^2}{r} + F_r
\end{gather}
In addition to the known variables from \autoref{eqn:ansys_conti} the viscosity $\mu$ and the swirl velocity $u_z$ play a role within the momentum equations.
Within all previous stated equations the product of $\nabla$ and $\mathbf{u}$ can be written as:

\begin{equation}
	\nabla \cdot \mathbf{u} = \dfrac{\partial u_x}{\partial x} + \dfrac{\partial u_r}{\partial r}+ \dfrac{u_r}{r}
\end{equation}
 

\subsection{energy equation}

\subsection{reaction equation}

\section{model setup}
\label{sec:mod_setup}

\subsection{geometry creation}

\subsection{meshing}

\subsection{setup}

\subsection{model evolution and refinement}

\subsection{model implementation}

\chapter{validation}
\label{chp:validation}

\section{experimental setup}

\section{experimental results}

\section{model results}

\section{comparison}

\end{document}