\documentclass[../thesis.tex]{subfiles}
\begin{document}

\addchap{Abstract}
\label{chp:abstract}

Within this work a model describing the development of a reaction-diffusion-advection front within a radial reactor is created and successfully implemented in \texttt{ANSYS FLUENT}. The model validation is done using existing experimental data. After assuring that the model shows physically believable behaviour, a parameter study is done using the dimensionless variables $Pe$ and $Sc$. The investigated variables are the front position, the fronts width and the total amount of product formed. These variables show a different behaviour dependant on the reactor's geometry and input variable values.
\newline

In dieser Arbeit wird ein Model zur Beschreibung von reaction-diffusion-advection Fronten für einen cylindrischen Reaktor entwickelt und in der Software \texttt{ANSYS FLUENT} implementiert. Das entwickelte Modell wird mit bestehenden experimentellen Daten verglichen um eine korrekte Funktionsweise zu gewährleisten. Anschließend wird eine Parameter Studie anhand der dimensionlosen Kennzahlen $Pe$ und $Sc$ durchgeführt. Es wird der Einfluss dieser Parameter auf die Position der Reaktionsfront sowie die Breite der Front als auch das ingesamt gebildete Produkt beschrieben und erklärt. Das Verhalten dieser Ausgangsgrößen zeigt eine starke Abhängigkeit von der Reaktor Geometrie sowie anderer veränderter Eingangsgrößen.

\end{document}