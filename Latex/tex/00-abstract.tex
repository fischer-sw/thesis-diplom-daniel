\documentclass[../thesis.tex]{subfiles}
\begin{document}

\addchap{Abstract}
\label{chp:abstract}

Within this work a parametric study investigating the influence of different parameters on an evolving Reaction-Diffusion-Advection front within a radial reactor is performed. A numerical model is created and implemented in \texttt{ANSYS FLUENT}. The model validation is done using existing experimental data. After assuring that the model shows physically believable behaviour, a parameter study is done using the dimensionless variables $Pe$ and $Sc$. The investigated variables are the front position, the fronts width and the total amount of product formed. These variables show a different behaviour, dependant on the reactor's geometry and input variable values.

\newpage

In dieser Arbeit wird der Einfluss einiger Eingangsgrößen auf eine sich bildende Reaction-Diffusion-Advection Front in einem zylindrischen Reaktor untersucht. Ein numerisches Modell wird dazu erstellt und in der Software \texttt{ANSYS FLUENT} implementiert. Das entwickelte Modell wird mit bestehenden experimentellen Daten verglichen, um eine korrekte Funktionsweise zu gewährleisten. Anschließend wird eine Parameter Studie anhand der dimensionslosen Kennzahlen $Pe$ und $Sc$ durchgeführt. Es wird der Einfluss dieser Parameter auf die Position der Reaktionsfront, sowie die Breite der Front, als auch das insgesamt gebildete Produkt beschrieben und erklärt. Das Verhalten dieser Ausgangsgrößen zeigt eine starke Abhängigkeit von der Reaktor Geometrie, sowie der veränderten Eingangsgrößen.

\end{document}